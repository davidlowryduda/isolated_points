\documentclass[11pt,reqno]{amsart}
% We ain't got no time for eq. nums. on the left

% Enable UTF-8 encodings for input, to enter é instead of \'{e}.
\usepackage[utf8]{inputenc}

\usepackage{amsmath,amsthm,amssymb}

% Presented to you by Technicolor, and the number 3
\usepackage{graphics}
\usepackage{hyperref}
\usepackage[usenames, dvipsnames]{xcolor}

% For full page usage, shockingly
\usepackage{fullpage}

% Don't worry about starred environments. YOU are the star!
\usepackage{mathtools}
\mathtoolsset{showonlyrefs}

% For ease in writing labels and references
%\usepackage{showkeys}
\usepackage[square,sort,comma,numbers]{natbib}

% For pretty hyperlinks
\definecolor{darkblue}{rgb}{0.0,0.0,0.3}
\hypersetup{colorlinks,breaklinks,
  linkcolor=darkblue,urlcolor=darkblue,
anchorcolor=darkblue,citecolor=darkblue}

\theoremstyle{plain}
\newtheorem{theorem}{Theorem}%[section]
\newtheorem*{theorem*}{Theorem}
\newtheorem{lemma}[theorem]{Lemma}
\newtheorem{proposition}[theorem]{Proposition}
\newtheorem*{proposition*}{Proposition}
\newtheorem{corollary}[theorem]{Corollary}
\newtheorem*{corollary*}{Corollary}
\newtheorem{claim}[theorem]{Claim}
\newtheorem{conjecture}[theorem]{Conjecture}
\newtheorem{question}[theorem]{Question}
\theoremstyle{definition}
\newtheorem{remark}[theorem]{Remark}
\newtheorem{definition}[theorem]{Definition}
\newtheorem{example}[theorem]{Example}
\newtheorem{exercise}[theorem]{Exercise}

%\numberwithin{equation}{section}

% Nongross real and imaginary parts
\renewcommand{\Im}{\operatorname{Im}}
\renewcommand{\Re}{\operatorname{Re}}
\DeclareMathOperator{\SL}{SL}
\DeclareMathOperator{\GL}{GL}
\DeclareMathOperator*{\Res}{Res}

\newcommand{\DLD}[1]{\textcolor{Plum}{DLD:\@ #1}}

% Don't use minted, b/c overleaf
% \usepackage{minted}
% \usemintedstyle{lovelace}
% \definecolor{mintbf}{gray}{0.95}
% \setminted{bgcolor=mintbf}

% Don't have subsections appear in TOC
%\setcounter{tocdepth}{1}

% This uses a hacky but reasonable parser that is not actually public
% \renewcommand{\MintedPygmentize}{./magma.py}

\title{Meeting notes}
\author{DLD}
\date{\today}

\begin{document}

\maketitle

We (Abbey, Travis, David, Sachi, and Himanshu) had a meeting
from today (29 March 2023) where we talked about
\texttt{NotIsolated.m}.
I annotated our code with notes from this meeting, which I show
here.

\vspace*{0.2cm}
\begin{quote}
  Pseudocode is the sort of code that compiles on a human, not on a computer.

  --- Travis
\end{quote}

This is all a description of \texttt{NotIsolated.m}. Note that this does
not include Filip's more recent enhancements --- that remains to be explained.

\section{\texttt{NotIsolated} function}\label{sec:notisolated}

The main function is \texttt{NotIsolated}, which is how we frame our code
examination.

\begin{verbatim}
/* First half of NotIsolated function, which determines if j
 * can't give isolated points. */
NotIsolated:=function(a, j);             // #1
    E:=EllipticCurve(a);
    G,n,S:=FindOpenImage(path, E);       // #2
    G0:=ReducedLevel(G);                 // #3 (Things to be proved!)
    G0t := sub<
      GL(2,Integers(#BaseRing(G0))) | [Transpose(g):g in Generators(G0)]
    >;                                   // #4
    k:=#BaseRing(G0t);
    good:=[];                            // "not isolated"
    bad:=[];                             // "potentially isolated"
    for b in Divisors(k) do
        if b gt 12 then                  // #5
            ww:=MinDegreeOfPoint(ChangeRing(G0t,Integers(b)));
                                         // #6
            // if the min degree >= g+1 then the dimension of the Reimann-Roch
            // space associated to the point of min degree is at least 2 hence it
            // is not sporadic.
            if  ww ge (Genus(Gamma1(b))+1) then
                Append(~good,<b, ww>);   // #7
            else
                // otherwise we need to check if gonality/degree reduction
                // arguments can be used
                Append(~bad,<b, ww>);
            end if;
        end if;
    end for;
\end{verbatim}


Along the side, I've indicated several annotation points. We address these in
turn. The numbers of each point in the following list align with an annotation
point in the code above.

\begin{enumerate}
  \item The function \texttt{NotIsolated} takes the coefficients \texttt{a =
  \{a\_1, \ldots, a\_6\}} of the Weierstrass equation of an elliptic curve and a
  $j$-invariant \texttt{j}. Ultimately, this function will try to show that $j$
  doesn't give an isolated point, and will either report that it was successful
  --- or that it couldn't show that $j$ doesn't give isolated points.

  Note the double negatives here. The algorithm confirms that an elliptic curve
  $E$ with $j$-invariant $j(E)$ \emph{doesn't} give sporadic points, or fails
  and gives partial information.

  \item \texttt{FindOpenImage} is Zywina's code. It returns three things: the
  image \texttt{G}, the index \texttt{n} of \texttt{G} in $\GL(2, \widehat{Z})$,
  and the intersection \texttt{S} of \texttt{G} with $\SL(2, \widehat{Z})$.

  \item\label{item:reduced_level_lemma} \texttt{ReducedLevel} is a function we
  define. There are many things to prove about it. I defer the individual
  description until \S\ref{sec:reducedlevel}. But the content of this line is
  the following:

  Zywina's code identifies an adelic level of the image of $G$. But to
  determine if a point is isolated, we don't need to consider the entire level.
  We can reduce the level to be more manageable. We (need to) prove that we can
  do this in Lemma~\ref{lemma:reduced_level} in \S\ref{sec:reducedlevel}.

  \item Note the transpose in this line! This confused us before. But this is
  because Magma things of vectors as row vectors, and thus matrices naturally
  act on the right.

  \item\label{item:b_geq_12} For each $b$, we assert that it is sufficient to
  consider only $b > 12$. We (need to) prove that this is sufficient in
  Lemma~\ref{lemma:b_geq_12}.
\end{enumerate}

% This Lemma corresponds to item~\ref{item:b_geq_12}.

\begin{lemma}\label{lemma:b_geq_12}
  (Roughly, we can restrict to $b > 12$).
\end{lemma}

\begin{enumerate}\setcounter{enumi}{5}
  \item\label{item:min_degree} \texttt{MinDegreeOfPoint} is a function we
  define. It returns the minimum degree of the field of definition of a point.
  We calculate this in a nice way, but I defer the individual description until
  \S\ref{sec:mindegreeofpoint}.

  \item\label{item:min_deg_genus} This is very simple, but we need a citation for the claim that if
  $\mathrm{mindeg}(b) \geq \mathrm{genus}(\Gamma_1(b)) + 1$, then the point is
  not isolated. We call this Lemma~\ref{lemma:min_deg_genus}.
\end{enumerate}

% This Lemma corresponds to item~\ref{item:min_deg_genus}.

\begin{lemma}\label{lemma:min_deg_genus}
  If $\deg(b) \geq \mathrm{genus}(\Gamma_1(b)) + 1$, then $b$ is not sporadic.
\end{lemma}


At the end of this half of the main function, we have accumulated two lists of
points, \texttt{good} and \texttt{bad}. Here, \emph{good} means that the point
is not isolated for the simple reasons from Lemma~\ref{lemma:min_deg_genus};
\emph{bad} simply means that we don't yet know.

The second half of the main function implements an idea based
on~\cite{Bourdon2019} to further eliminate points.


\begin{verbatim}
    /* continuing from before */
    // Refuting levels and degrees based on the level reduction theorem of BELOV.
    // If j is a sporadic point of degree d in level m then it becomes a point of
    // d/deg(f) in level n where f:X1(m)-->X1(n) is the natural projection map.
    remove:={};                          // #8 START
    for x in bad do
        for y in good do
            if IsDivisibleBy(x[1],y[1]) then
                b:=x[1] div y[1];
                deg:=b^2*&*[Rationals()
                              | 1-1/p^2 : p in PrimeDivisors(x[1])
                              | p notin PrimeDivisors(y[1]) ];
                if deg eq x[2] div y[2] then Include(~remove,x); end if;
            end if;
        end for;
    end for;                             // #8 END

    // Now we check using gonality arguments if the levels, degrees in the set
    // bad but not in remove can be handled using gonality arguments.
    bad := SequenceToSet(bad);
    if bad ne remove then                // #9
        supbad := bad diff remove;
        return [*j, a, false, supbad*]; //return j-invariant we already have
    end if;
    return [*j, a, true*];
end function;
\end{verbatim}

This code does two things.

The region surrounded by \texttt{\#8 START} and
\texttt{\#8 END} is an implementation of~\cite{Bourdon2019}.

\begin{enumerate}\setcounter{enumi}{7}
  \item\label{item:belov} The region surrounded by \texttt{\#8 START} and
  \texttt{\#8 END} is an implementation of~\cite{Bourdon2019}. Conceptually, we
  can think of this region as a function (creating \texttt{remove} from
  \texttt{good} and \texttt{bad}) that is a concrete code function on its own.
  We describe this implementation further in Proposition~\ref{prop:belov}.

  \item Finally, we see if there is anything left after trying to remove
  as much as possible using Proposition~\ref{prop:belov}. If there are, then we
  return the remaining elements of uncertainty (which we call \texttt{supbad},
  presumably short for \emph{super bad}).
\end{enumerate}

\begin{proposition}\label{prop:belov}
  The algorithm in \#8 correctly determines \ldots
\end{proposition}

\begin{theorem}\label{thm:main}
  The algorithm in \texttt{NotIsolated} either confirms that a $j$-invariant doesn't
  give rise to sporadic points or indicates that additional information is
  necessary. (There is a lot to be said here, and it will incorporate each
  other intermediate result).
\end{theorem}

\clearpage
\section{\texttt{ReducedLevel} function}\label{sec:reducedlevel}

We now consider \texttt{ReducedLevel}, which is a rather important piece of
code. Despite how quickly we wrote it (almost entirely written in the first
day), a lot of things are happening.

\begin{verbatim}
/* NOTE: We explicitly check those primes dividing m when G doesn't surject to
 * GL2FP. */
NonSurjectivePrimes:=function(G)                     // #1
    m:=Modulus(BaseRing(G));
    return [p:p in PrimeFactors(m)|#ChangeRing(G,GF(p)) ne #GL(2,GF(p))];
end function;

// Level reduction from the output of Zywina's algorithm
ReducedLevel:=function(G)
    m:=Modulus(BaseRing(G));
    NS:=Set(NonSurjectivePrimes(G));
    sE:={2,3} join NS;                    // always consider 2 and 3
    m0:=&*[p^Valuation(m,p):p in sE];
    G:=ChangeRing(G,Integers(m0));
    for p in PrimeFactors(m0) do                     // #2
        while Valuation(m0,p) gt 1                   // #3
                and #G/#ChangeRing(G,Integers(m0 div p)) eq p^4 do
            m0:=m0 div p;
            G:=ChangeRing(G,Integers(m0));
        end while;
        if not p in NS and Valuation(m0,p) eq 1      // #4
               and #G/#ChangeRing(G,Integers(m0 div p)) eq #GL(2,GF(p)) then
            m0:=m0 div p;
            G:=ChangeRing(G,Integers(m0));
        end if;
    end for;
    return G,m0;
end function;
\end{verbatim}

Along the side, I've indicated several annotation points. We address these in
turn. The numbers of each point in the following list align with an annotation
point in the code above.

\begin{enumerate}
  \item \texttt{NonSurjectivePrimes} is almost pseudocode as is, representing
  almost exactly a definition.

  \item In our meeting, we had a couple of thoughts about the relationship
  between \texttt{NS}, \texttt{sE}, \texttt{m0}, and
  \texttt{PrimeFactors(m0)}. We briefly talked about whether it was inefficient
  to compute the prime factors of \texttt{m0} again in this line, as in some
  sense we already know the primes from just above.

  \item The moral story of this line is that we check to make sure that the
  index when passing from $G(m_0)$ to $G(m_0/p)$ is as large as possible. If it
  is, then nothing is lost by removing a power of $p$.

  We had a lot of discussion about this when $p = 2$. See Proposition 3.5
  of~\cite{Bourdon2019} (or other places) for the idea. But note that $2$ needs
  to be handled differently in principle.

  \item The purpose of this line is to possibly consider removing the factors
  of $2$ and $3$ that we (forcibly) added above.
  Notice that \texttt{p in PrimeFactors(m0)} and \texttt{not p in NS} implies
  that $p \in \{2, 3\}$.

  It might be possible that this portion of code could be simplified.
\end{enumerate}

The primary claim about what this code for is contained in the Level Reduction
Lemma, Lemma~\ref{lemma:reduced_level}. But we need to guarantee that this
implementation is correct.

\begin{lemma}\label{lemma:reduced_level}[Level Reduction Lemma]
  Suppose that $E/\mathbb{Q}$ has adelic image $G$ in $\GL(2, \widehat{Z})$ and
  $G$ has level $M$. To determine if $E$ gives rise to a sporadic, it suffices
  to consider level $N$ \ldots (where $N$ consists essentially of nonsurjective
  primes, and also $2$ and $3$).
\end{lemma}


\begin{proposition}
  The algorithm in \texttt{ReducedLevel} correctly reduces level.
\end{proposition}

I note that this will probably use Proposition 3.5 of~\cite{Bourdon2019} (or
other places) for the idea, and should pay close attention to what happens for
the prime $2$.



\section{\texttt{MinDegreeOfPoint} function}\label{sec:mindegreeofpoint}

We now consider \texttt{MinDegreeOfPoint}.

\begin{verbatim}
VectorOrder:=function(v)
    m:=#Parent(v[1]);
    g:=GCD([m, Integers()!v[2], Integers()!v[1]]);
    return m div g;
end function;

//compute min degree of a point i.e. the min degree of the field of definition
//of a point expressed as a 1 x 2  vector.
MinDegreeOfPoint:=function(G)
    m:=Modulus(BaseRing(G));
    H:=sub<GL(2,Integers(m))|G,-G!1>;           // #1
    orb:=Orbits(H);                             // #2
    sorb:=Sort(orb,func<o1,o2|#o1-#o2>);        // #3
    s2:=[x: x in sorb| VectorOrder(x[1]) eq m]; // #4
    return (#s2[1]) div 2;                      // #5
end function;
\end{verbatim}

Along the side, I've indicated several annotation points. We address these in
turn. The numbers of each point in the following list align with an annotation
point in the code above.

\begin{enumerate}
  \item When finding the minimum degree of a point, we deliberately include
  $-1$ in the group.

  \item \texttt{Orbits} is a standard magma command that outputs the orbits as
  a list of lists of row-vectors. It's very fast, and led to radical speed
  improvements when we began to use it instead of our earlier code.

  \item This defines a sorting function. The effect is that \texttt{sorb}
  contains sorted orbits, sorted in increasing order of size.

  \item We isolate to those elements with compatible \texttt{VectorOrder}s.

  \item As we included $-1$, we can always divide by $2$ when we return the
  elements. Since the orbits are sorted, the first element will be the
  smallest, which is why we return the first element.
\end{enumerate}

I think there are perhaps two things to show here.

\begin{lemma}
  The algorithm in \texttt{VectorOrder} correctly returns \ldots
\end{lemma}

\begin{proposition}
  The algorithm in \texttt{MinDegreeOfPoint} correctly returns the minimum
  degree of the field of definition.

  Either in this proposition or in an additional lemma, describe why we can
  divide the resulting degree by $2$ \emph{always}, including when $-1 \in G$
  already. \DLD{I only note this because I can't remember why this is true.}
\end{proposition}


\section{Additional notes}

In the process of writing these descriptions, we will probably rewrite parts of
the code. As we're going to make the code public, we will also need to make a
\emph{sanitized} version that we stand by.

There is also a consideration of what we're phrasing. Initially, we thought
about sporadic $j$-invariants. But we're moving on from that.

Finally, there are additional components of processing that we've done but
which aren't incorporated here. These include

\begin{enumerate}
  \item The basic gonality processing, and
  \item The other things that Filip wrote about in zulip.
\end{enumerate}


\vspace{20 mm}
\bibliographystyle{alpha}
\bibliography{bibliofile}

\end{document}
