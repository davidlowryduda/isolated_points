\documentclass[11pt,reqno]{amsart}
% We ain't got no time for eq. nums. on the left

% Enable UTF-8 encodings for input, to enter é instead of \'{e}.
\usepackage[utf8]{inputenc}

\usepackage{amsmath,amsthm,amssymb}
\usepackage[ruled,vlined,linesnumbered]{algorithm2e}

\usepackage{colonequals}

% Presented to you by Technicolor, and the number 3
\usepackage{graphics}
\usepackage{hyperref}
\usepackage[usenames, dvipsnames]{xcolor}

% For full page usage, shockingly
\usepackage{fullpage}

% Don't worry about starred environments. YOU are the star!
\usepackage{mathtools}
\mathtoolsset{showonlyrefs}

% For ease in writing labels and references
%\usepackage{showkeys}
\usepackage[square,sort,comma,numbers]{natbib}

% For pretty hyperlinks (I changed this because I cannot see what is a ref/link and what is not, sorry! - Sachi)
\definecolor{darkblue}{rgb}{0.0,0.0,1}
\hypersetup{colorlinks,breaklinks,
  linkcolor=darkblue,urlcolor=darkblue,
anchorcolor=darkblue,citecolor=darkblue}

\theoremstyle{plain}
\newtheorem{theorem}{Theorem}%[section]
\newtheorem*{theorem*}{Theorem}
\newtheorem{lemma}[theorem]{Lemma}
\newtheorem{proposition}[theorem]{Proposition}
\newtheorem*{proposition*}{Proposition}
\newtheorem{corollary}[theorem]{Corollary}
\newtheorem*{corollary*}{Corollary}
\newtheorem{claim}[theorem]{Claim}
\newtheorem{conjecture}[theorem]{Conjecture}
\newtheorem{question}[theorem]{Question}
\theoremstyle{definition}
\newtheorem{remark}[theorem]{Remark}
\newtheorem{definition}[theorem]{Definition}
\newtheorem{example}[theorem]{Example}
\newtheorem{exercise}[theorem]{Exercise}
% \newtheorem{algorithm}[theorem]{Algorithm}
\DeclareMathOperator{\gon}{gon}
%\numberwithin{equation}{section}

\renewcommand{\epsilon}{\varepsilon}
\renewcommand{\phi}{\varphi}
\renewcommand{\theta}{\vartheta}
\newcommand{\Supp}{\operatorname{Supp}}
\renewcommand\labelenumi{(\theenumi)}
\renewcommand{\theenumi}{\roman{enumi}}

%%% Basic Macro%%%%%%%%%%%%%%%%%%%%%%%%%%%%%%%%%%%%%%%%
\def\ol#1{\overline{#1}}% 		overline
\def\wh#1{\widehat{#1}}% 	wide hat
\def\wt#1{\widetilde{#1}}% 	wide tilde
\def\ul#1{\underline{#1}}% 	underline
\def\smcompactification#1{\ol{#1}}% 	wide tilde
%%% Define \Alphabet&\endpiece------------------------------------------------------------------------
\def\Alphabet{A,B,C,D,E,F,G,H,I,J,K,L,M,N,O,P,Q,R,S,T,U,V,W,X,Y,Z}%  Capitalized Alphabet
\def\alphabet{a,b,c,d,e,f,g,h,i,j,k,l,m,n,o,p,q,r,s,t,u,v,w,x,y,z}%	lowercase alphabet
\def\endpiece{xxx}%									marks end of list
%%% Define \makeAlphabet------------------------------------------------------------------------
\def\makeAlphabet[#1]{\expandafter\makeA#1,xxx,}%		Ex. \makeAlphabet[A,B]
\def\makealphabet[#1]{\expandafter\makea#1,xxx,}%		Ex. \makealphabet[c,d]
\def\makeA#1,{\def\temp{#1}\ifx\temp\endpiece\else%
	\mkbb{#1}\mkfrak{#1}\mkbf{#1}\mkcal{#1}\mkscr{#1}\mkbs{#1}\expandafter\makeA\fi}%
\def\makea#1,{\def\temp{#1}\ifx\temp\endpiece\else\mkfrak{#1}\mkbf{#1}\mkbs{#1}\expandafter\makea\fi}%
\def\mkbb#1{\expandafter\def\csname bb#1\endcsname{\mathbb{#1}}}%      Define bb
\def\mkfrak#1{\expandafter\def\csname fr#1\endcsname{\mathfrak{#1}}}%    Define frak
\def\mkbf#1{\expandafter\def\csname b#1\endcsname{\mathbf{#1}}}%           Define bold letters
\def\mkcal#1{\expandafter\def\csname c#1\endcsname{\mathcal{#1}}}%       Define calligraphy
\def\mkscr#1{\expandafter\def\csname s#1\endcsname{\mathscr{#1}}}%       Define script
\def\mkbs#1{\expandafter\def\csname bs#1\endcsname{{\boldsymbol{#1}}}}%       Define bold symbol
%%% Define \makeop-------------------------------------------------------------------------------------------------------
\def\makeop[#1]{\xmakeop#1,xxx,}%					Ex. \makeop[Hom,Spec]
\def\mkop#1{\expandafter\def\csname #1\endcsname{{\mathrm{#1}}}} %
\def\xmakeop#1,{\def\temp{#1}\ifx\temp\endpiece\else\mkop{#1}\expandafter\xmakeop\fi}%
\def\makeup[#1]{\xmakeup#1,xxx,}%					Ex. \makeop[Hom,Spec]
\def\mkup#1{\expandafter\def\csname #1\endcsname{{\mathrm{#1}\,}}} %
\def\xmakeup#1,{\def\temp{#1}\ifx\temp\endpiece\else\mkup{#1}\expandafter\xmakeup\fi}%
%%% Initialize------------------------------------------------------------------------------------------------------------------
% Define Alphabets.  Alphabets stored in \Alphabet
\makeAlphabet[\Alphabet]%				Define bb, frak, bf, cal for Capitalized Alphabet
\makealphabet[\alphabet]%  				Define frak and bf for uncapitalized alphabet
% Define Operators.  Separate Items by using comma.
\makeop[Hom,Aut,End,Mor,SL,GL,H,ord,Irr,Ell,Gal,Cl,Pic,NS,Gal,d,Re,Im,res,Symb,Ev,Char,Ram,SU]
% 		Homs
\makeup[Spec,Proj,id,dR,new,old,AJ,tr,dim,ker,im,coker]

% Nongross real and imaginary parts
\renewcommand{\Im}{\operatorname{Im}}
\renewcommand{\Re}{\operatorname{Re}}
\newcommand{\Q}{\bQ}
\newcommand{\Z}{\bZ}
\newcommand{\PP}{\mathbb P}
\newcommand{\filip}[1]{{\color{cyan} \textsf{$\sun\sun\sun$ Filip: [#1]}}}
\newcommand{\davidnote}[1]{\textcolor{Plum}{David: #1}}

\newcommand{\abbey}[1]{\textcolor{blue}{Abbey: #1}}
\newcommand{\dld}[1]{\textcolor{Plum}{dld: #1}}
\newcommand{\sachi}[1]{\textcolor{purple}{Sachi: #1}}
\newcommand{\timo}[1]{\textcolor{red}{Timo: #1}}
% \newcommand{\travis}[1]{\textcolor{chartreuse}{Travis: #1}}

% Don't have subsections appear in TOC
%\setcounter{tocdepth}{1}

\title{Towards a Classification of Sporadic $j$-invariants}
\author[Doe]{Your Name Here}
\address{
  John Doe,
  Monsters University,
  Department of Mathematics,
  123 Main Street,
  Anywhere, CA
  United States of America
}
\email{\url{foobar@example.com}}
\thanks{JD is supported by the Imaginary Friend association}

\author[Bourdon]{Abbey Bourdon}
\address{
  Abbey Bourdon,
  Wake Forest University,
  Department of Mathematics, 127 Manchester Hall, PO Box 7388, Winston-Salem, NC 27109
}
\email{\url{bourdoam@wfu.edu}}
\thanks{AB is supported by NSF Grant DMS-2145270.}

\author[Hashimoto]{Sachi Hashimoto}
\address{%
  Sachi Hashimoto,
  Max Planck Institut für Mathematik in den Naturwissenschaften,
  Inselstraße 22,
  04103 Leipzig
}
\email{\url{sachi.hashimoto@mis.mpi.de}}
\urladdr{\url{sachihashimoto.github.io/}}


\author[Keller]{Timo Keller}
\address{Timo Keller, Leibniz Universität Hannover, Institut für Algebra, Zahlentheorie und Diskrete Mathematik, Welfengarten 1, 30167 Hannover, Germany}
\email{keller@math.uni-hannover.de}
\urladdr{\url{https://www.timo-keller.de}}


\author[Lowry-Duda]{David Lowry-Duda}
\address{%
  David Lowry-Duda, ICERM, 121 South Main Street, Box E, 11th Floor,
  Providence, RI, 02903
}
\email{\url{david@lowryduda.com}}
\urladdr{\url{https://davidlowryduda.com}}
\thanks{%
  DLD was supported by the Simons Collaboration in Arithmetic Geometry, Number
  Theory, and Computation via the Simons Foundation grant 546235.
}

\author[Shukla]{Himanshu Shukla}
\address{Himanshu Shukla, Mathematisches Institut, Uiversit\"{a}t Bayreuth, Universit\"{a}tstrasse 30, 95444 Bayreuth, Germany}
\email{Himanshu.Shukla@uni-bayreuth.de}
\urladdr{\url{https://www.mathe2.uni-bayreuth.de/hishukla/}}
\thanks{HS is supported by the DFG-grant STO 299/17-1}

\date{\today}

\begin{document}
\begin{abstract}
We develop an algorithm to test whether a non-CM elliptic curve $E/\Q$ gives rise to a sporadic point of any degree on any modular curve of the form $X_1(N)$. This builds on prior work of Zywina which gives a method for computing the image of the adelic Galois representation associated to $E$. Running this algorithm on all elliptic curves presetly in the LMFDB yields strong evidence for the fact that $E$ gives rise to a sporadic point if and only if $j(E)=-140625/8$ or $-9317$.
    \end{abstract}
\maketitle


The modular curve $X_1(N)$ is an algebraic curve over $\Q$ whose non-cuspidal points parametrize elliptic curves with a distinguished point of order $N$. We are interested in studying \textbf{sporadic} points $x \in X_1(N)$, which are points for which there are only finitely many points on $X_1(N)$ of degree at most $\deg(x)$. Hence a non-cuspidal sporadic point corresponds to an elliptic curve with a point of order $N$ in ``usually low degree." Elliptic curves with complex multiplication (CM) provide many natural examples of sporadic points, as the extra endomorphisms of a CM elliptic curve give constraints on the size of the image of the associated Galois representation. Indeed, there exist sporadic CM points on $X_1(N)$ for all $N\geq 721$; see \cite[Theorem 8.2]{CGPS2022}.

Non-CM sporadic points on $X_1(N)$ remain much more mysterious. One recent line of investigation has focused on the class of sporadic points associated to non-CM elliptic curves with $j$-invariant in $\Q$. To date, there are only two known examples of such curves, up to isomorphism over $\overline{\Q}$:
\begin{itemize}
\item The elliptic curve with $j$-invariant $-140625/8$ corresponds to points of degree 3 on $X_1(21)$. As there are only finitely many points of degree at most 3 on this modular curve, these are sporadic. This example was first discovered by Najman \cite{najman16}. In fact, this is the unique elliptic curve giving a sporadic point of degree at most 3 on \emph{any} modular curve of the form $X_1(N)$, as shown in recent work of Derickx, Etropolski, van Hoeij, Morrow, and Zureick-Brown \cite{DEvHMZB2021}.
\item The elliptic curve with $j$-invariant $-9317$ gives a degree 6 point on $X_1(37)$, as in work of van Hoeij \cite{vanHoeij}. Since this degree is less than half the $\Q$-gonality of $X_1(37)$, as computed in \cite{DerickxVanHoeij2014}, the point is necessarily sporadic by work of Frey \cite{frey}.
\end{itemize}

We say these are \textbf{sporadic $j$-invariants} since they are the image of a sporadic point on $X_1(N)$. We have good reason to believe that the collection of all sporadic $j$-invariants in $\Q$ is finite.
\begin{theorem}[Bourdon, Ejder, Liu, Odumodu, Viray \cite{BELOV}]
Suppose there exists a constant $C$ such that the mod $\ell$ Galois representation of any non-CM elliptic curve over $\Q$ is surjective for primes $\ell>C$. Then there are only finitely many sporadic $j$-invariants in $\Q$.
\end{theorem}

\noindent The assumption about Galois representations stated in Theorem 1 was originally asked as a question by Serre \cite{serre72}, and it has now been formally conjectured by both Sutherland \cite{sutherland} and Zywina \cite{ZywinaImages}. It is even suspected that $C=37$. Theorem 1 is known unconditionally for points of odd degree \cite{OddDeg}: indeed, $j=-140625/8$ is the only non-CM $j$-invariant in $\Q$ giving a sporadic point of odd degree on $X_1(N)$, even as $N$ ranges over all positive integers.

In \cite{BELOV}, they ask whether one can explicitly determine the set of sporadic $j$-invariants in $\Q$, and this is the motivation for the present work. In particular, we conjecture that the two non-CM sporadic $j$-invariants identified above are in fact the \emph{only} sporadic $j$-invariants in $\Q$ associated to non-CM elliptic curves.

\begin{conjecture}
If $x\in X_1(N)$ is a sporadic point with $j(x) \in \Q$, then $j(x)=-140625/8$ or $-9317$ or is one of the 13 CM $j$-invariants in $\Q$.
\end{conjecture}

\noindent Since any CM elliptic curve is known to produce sporadic points on infinitely many modular curves of the form $X_1(N)$, it follows conversely that every $j$-invariant in this set is sporadic.

The basis for this conjecture is the following result.

\begin{theorem}
Let $x=[E,P]\in X_1(N)$ be a non-CM sporadic point with $j(E) \in \Q$.
Fix an equation for $E/\Q$ and let $N_E$ denote its conductor.
Suppose that one of the following holds:
\begin{itemize}
    \item$N_E \leq 500{,}000$,
    \item $N_E$ is only divisible by primes $p \leq 7$, or
    \item $N_E=p \leq 200{,}000{,}000$ for some prime number $p$.
\end{itemize}
Then $j(E) =-140625/8$ or $-9317$.
\end{theorem}

Since both known examples of sporadic points on $X_1(N)$ associated to non-CM rational $j$-invariants lie above exceptional rational points on $X_0(N)$ -- that is, they correspond to rational points in cases where the set of all such rational points is finite -- is natural to ask whether this construction might yield other examples of sporadic points. Our work shows that the answer is no.
\begin{corollary}
Let $X_0(N)$ have genus greater than 0, and let $E$ be a non-CM elliptic curve corresponding to a rational point on $X_0(N)$. If there exists a sporadic point $x \in X_1(N')$ with $j(x)=j(E)$, then $j(E) =-140625/8$ or $-9317$.
\end{corollary}

\section{Background}

\begin{theorem}
Let $E/\Q$ be a non-CM elliptic curve, and let $\rho_E$ denote the adelic Galois representation of $E$. Let $M_E$ be any positive integer such that
\[
\im \rho_E=\pi^{-1}(\im \rho_{E, M_E}).
\]
Then for any $x \in X_1(n)$ with $j(x)=j(E)$ we have $\deg(x)=\deg(f)\cdot \deg(f(x))$, where $f: X_1(n) \rightarrow X_1(\gcd(n,M_E))$ denotes the natural map. It follows that:
\begin{enumerate}
\item If $x$ is $\mathbf{P}^1$-isolated, then $f(x)$ is $\mathbf{P}^1$-isolated.
\item If $x$ is AV-isolated, then $f(x)$ is AV-isolated.
\item If $x$ is sporadic, then $f(x)$ is sporadic.
\end{enumerate}
\end{theorem}

\begin{proof}
This follows from the arguments \cite[$\S5.3$]{BELOV} and \cite[Theorem 4.3]{BELOV}. \abbey{This is mentioned in Remark 5.5 of \cite{BELOV} in particular. I'm not sure if we should include more details here.}
\end{proof}

\section{Proving that a point is not sporadic}
The following algorithm is the main procedure for determining a $j$-invariant is not sporadic. It gives an overview of the structure.

\begin{algorithm}[h!]
\caption{Main Algorithm}\label{alg:mainalgorithm}
\KwIn{A non-CM $j$-invariant $j = j(E) \in \Q$ and $a$-invariants for $E$ with $j$-invariant $j$.}
\KwOut{Finite list of pairs $(n,d)$ such that any isolated point on $X_1(N)$ associated to $E$ maps down under the natural projection map to an isolated point of degree $d$ on $X_1(n)$ in the list. }
Compute the adelic image $G$ of $E/\Q$ as a subgroup of $\GL_2(\hat{\Z})$ using Zywina's algorithm. Represent the output as some \emph{level} $N$ together with a subgroup $G$ of $\GL_2(\Z/N\Z)$\;
Using Algorithm \ref{alg:reduce_level}, reduce the level to $G, m_0$\; 
Define $L \colonequals \{\}$\;
\For{all $12 < b \mid m_0$}{
Using Algorithm~\ref{alg:compute_degree} compute the set $D$ of degrees of all fields of definition of a point with respect to $G \mod b$, the image of $G$ in $\GL_2(\Z/b\Z)$\;
Set $L \colonequals L \cup \{ (b, d) : d \in D\}$ \;}
Using Algorithm \ref{alg:level_mapping} filter the list $L$ to obtain a refined list $L'$\;
\Return{the list $L'$ } \;
    %\filip{We need to look at the orbits of $\langle G, -G \rangle \mod b$ on $(\Z/b\Z)^2$, not $\bP^1(\Z/b\Z)$. If we consider the action on $\bP^1(\Z/b\Z)$, the orbits might be too small and hence the degree we get. Furthermore, it is necessary to consider only the elements of $(\Z/b\Z)^2$ of order $b$.}
\end{algorithm}



\begin{theorem}
 Algorithm \ref{alg:mainalgorithm} is correct. 
\end{theorem}
\begin{corollary}
If Algorithm \ref{alg:mainalgorithm} outputs an empty set on some $j$-invariant $j = j(E)$ and $a$-invariants $a$ for $E$, then $j$ is not a  $\mathbf{P}^1$-isolated point on $X_1(N)$.
\end{corollary}
\abbey{Zywina said the results returned from his adelic image algrithm are guaranteed to be correct and that ``errors will always occur if $E$ gives rise to an unknown exceptional rational point on certain high genus modular curves." Since an error in his algorithm will result in an error in our algorithm, I don't know what should be said about this in our writeup.} \sachi{maybe the easiest thing is just to add a remark saying that the results are guaranteed to be correct, and getting an error instead of a result is possible but that we did not encounter it?} \timo{I second that.} \abbey{alternatively, separate Zywina's algorithm from ours in this paper}

(In the \texttt{Magma} implementation, we have to be careful that \texttt{Magma} uses right actions, but in our mathematics, we use left action. We therefore use the transposes of the matrices in $\GL_2(\Z/N\Z)$.)


\section{Reduce Level}

\abbey{Travis,Sachi, Abbey}

Using Zywina's algorithm for the level of the adelic Galois representation as a starting point, we may obtain the level of the $m$-adic Galois representation associated to $E$ as in Theorem \ref{BELOVthm}.


\begin{algorithm}[H]
  \KwIn{$G\subseteq \GL_2(\Z/N\Z)$ such that $\im\rho_E=\pi^{-1}(G)$ and $N$ is the level}
  \KwOut{$G\subseteq \GL_2(\Z/m_0\Z)$ such that $G=\im\rho_{E,m_0}$ and $\im\rho_{E,m^{\infty}}=\pi^{-1}(\im\rho_{E,m_0})$ where $m$ is the product of the non-surjective primes and $2$ and $3$ (if they do not already divide $m$), and $m_0$ is minimal }
  \caption{Reduce level}%
  \label{alg:reduce_level}
  Let $P_E = \{\ell|N \text{ prime} \colon \rho_{E,\ell} \text{ is non-surjective} \}$\;
  Let $S_E= P_E \cup \{ 2, 3\}$\;
  Let $m_0 = \prod_{\ell \in S_E} \ell^{v_{\ell}(N)}$\;
  \For{$p \in \{ \ell | m_0 \colon \ell \text{ prime} \}$}{
  \While{$p |m_0$ and $\#G/\#(G \pmod{m_0/p}) = p^4$}{
    $m_0 = m_0/p$\;
    $G = G \pmod{m_0}$\;
  }
  \If{not $p$ in $P_E$ and $\ord_p(m_0) = 1$ and $\#G/\#(G \pmod{m_0/p}) = \#\GL_2(\Z/p\Z)$}{
  $m_0 = m_0/p$\;
   $G = G \pmod{m_0}$\;}}
  \Return{$G,m_0$}
\end{algorithm}

  \begin{proposition}
    Algorithm~\ref{alg:reduce_level} is correct.
  \end{proposition}
  \begin{proof}
    Since $N$ is the level of the adelic representation and both $m$ and $m_0$ divides $N$, we must have that $\im\rho_{E, m^\infty} = \pi^{-1}(\im \rho_{E,m_0})$. This property is an invariant of the for-loop. The for-loop
    terminates once as many powers of $\ell$ have been removed from $m_0$
    as possible while maintaining this invariant. Thus $m_0$ is the minimal such integer and is thus the level.
  \end{proof}

\abbey{give example}


\section{Compute Degrees}

The following algorithm computes the degrees of closed points on $X_1(N)$ associated to $E$.

% \noindent \textbf{Input:} $\mathcal{I}_1$ from step 1 \\
% \textbf{Output:} For each $d \in \mathcal{I}_1$, compute degree(s) of closed point on $X_1(N)$ associated to $E$. Return ??? \abbey{update algorithm description}

\begin{algorithm}[H] \caption{Compute Degree}\label{alg:compute_degree}
  \KwIn{$G\subseteq \GL_2(\Z/N\Z)$ such that $\im\rho_E=\pi^{-1}(G)$ and $N$ is the level}
  \KwOut{all degrees of $x$-coordinates of points}
  Let $H \colonequals \langle G, -I_2\rangle \leq \GL_2(\Z/N\Z)$\;
  Compute the orbits of $H$ acting on $(\Z/N\Z)^2$\label{Inc-1}\;
  Filter the orbit representatives for orbits that have order equal to $N$\;
  Compute the orbit sizes $s$ and ``degrees'' as $s/2$ \label{DivBy2}\;
  Remove duplicates\label{RemDup}\;
  \Return{degrees}
\end{algorithm}

Note that only working with orbit(s) (representatives) significantly speeds up the implementation. The computation of the orbits of $G$ acting on $(\Z/N\Z)^2$ is efficiently implemented in Magma.

\begin{proposition}
Let $E/\Q$ be a non-CM elliptic curve and $N \in \Z_{>0}$. If $\im \rho_{E,N} \cong G \leq \GL_2(\Z/N\Z)$, then the Algorithm \ref{alg:compute_degree} outputs the sequence consisting of all possible degrees of $x(P)$, where $P\in E[N]$ is a point of order $N$.
\end{proposition}
\begin{proof}
    We begin by noting that $(E, P)$ and $(E, -P)$ \abbey{induce the same closed} point on $X_1(N)$, for $P \in E[N]$. Therefore, the degree of
    a point represented by $(E, P)$, when
    $E$ is defined over $\Q$ depends only on $x(P)$.
    If $E[N]$ is identified with $(\Z/N\Z)^2$, then $\deg(P)=\# (G. P)$, where $P$ on the left hand side is considered as an element of $(\Z/N\Z)^2$
    and on right hand side as an element of $E[N]$.
    If $P$ and $-P$ are in the same Galois orbit, i.e., $-I_2\in G$, then $[\Q(x(P)):\Q]=[\Q(P):\Q]/2$. On the other hand, one has $[\Q(x(P)):\Q]=[\Q(P):\Q]$, when
    $-I_2\notin G$.
    Therefore, computing the orbit sizes with respect to the
    group $H\coloneqq \langle G, -I_2 \rangle$ (step~\ref{Inc-1} in Algorithm \ref{alg:compute_degree}) and dividing
    by $2$ (step~\ref{DivBy2}) handles both the cases, because in the later case (i.e., when $-I_2\notin G$) the orbit sizes will be doubled.
    Finally, step~\ref{RemDup} makes sure that every degree occurs only once in the output sequence.
\end{proof}


\section{Filtering by Level Mapping}

\begin{theorem}[\cite{BELOV}]\label{BELOVthm}
Let $E/\Q$ be a non-CM elliptic curve, and let $m$ be the product of 2, 3, and all primes $\ell$ where the $\ell$-adic Galois representation of $E$ is not surjective. Let $m_0$ be the level of the $m$-adic Galois representation associated to $E$ and let $f: X_1(n) \rightarrow X_1(\gcd(n,m_0))$ denote the natural map. If $x\in X_1(n)$ is an isolated point with $j(x)=j(E)$, then $f(x)\in X_1(\gcd(n,m_0))$ is an isolated point.
\end{theorem}

\dld{This is \texttt{FilterByLevelMapping}, which used to be called
\texttt{RefuteLevel}. It acts as a filter. We feed in a list, and the output is
a (hopefully smaller) list guaranteed to contain all isolated points. If the
list is empty, then we have shown that the curve doesn't give rise to an
isolated point! If not, then perhaps other work is necessary.}

\dld{I imagine other filters being potentially applied afterwards, each
removing some set of non-isolated points.}

\begin{algorithm}[H]\caption{Filter by Level Mapping}\label{alg:level_mapping}
  \KwIn{A list of pairs $(N, d)$ such that $E$ gives a non-CM point on $X_1(N)$
  of degree $d$.}
  \KwOut{A list of pairs $(N, d)$ containing all isolated points.}
  (sketch)\;
  create list of guaranteed nonisolated points using easy Riemann--Roch\;
  filter potentially isolated points via Theorem~\ref{BELOVthm}\;
  \Return{remaining $(N, d)$ pairs.}
\end{algorithm}

\dld{I collect the ingredients for use. The easy Riemann--Roch argument we use
is from the following lemma. In principle, we could use any source of
guaranteed nonisolated points.}

\abbey{give example}

\begin{lemma}
Let $C$ be a curve over a number field $k$ with genus $g$. If $x \in C(\overline{k})$ is a closed point with degree
$\geq g+1$, then $x$ is not $\mathbb{P}^{1}$-isolated.
\end{lemma}

\begin{proof}
This is Lemma 13 in revision of B., Gill, Rouse, Watson. \cite{OddDeg}
\end{proof}

\dld{For the filter, we apply (the contrapositive of) Theorem~\ref{BELOVthm}.
We also use the degree of the natural map as in Proposition~2.2 of BELOV.}

\section{Remaining Filters}
\abbey{Filip? Abbey?}
Filip's arguments.



\section{Genus 0 adelic images do not produce sporadic points}

Let $E/ \Q$ be an elliptic curve and $G\in \GL_2(\hat \Z)$ its adelic image. By $G(n)$ denote the image of its mod $n$ representation. Denote by $B_1(n)$ the subgroup of $\GL_2(\Z/n\Z)$ consisting of the upper triangular matrices with a $1$ in the upper left entry. Note that $X_{B_1(n)}=X_1(n)$. We say that a congruence group $\Gamma$ is of genus $g$ if $X_\Gamma$ is of genus $g$. We say that a point $P$ corresponds to an elliptic curve $E$ if $j(P)=j(E)$.

\begin{proposition}
Let $E/\Q$ be an elliptic curve with adelic image of genus $0$. Then there does not exist an $n$ such that a point corresponding to $E$ on $X_1(n)$ is sporadic.
\end{proposition}
\begin{proof}
Since $G$ is by assumption of genus $0$, it follows that so is $G(n)$. Let $P$ be a $\Q$-rational point on $X_{G(n)}$ corresponding to $E$.

%Let $B$ be the group generated by $G(n)$ and a conjugate of $B_1(n)$ in $\GL_2(\Z/n\Z)$; this is again a group of genus $0$ on which the image $P'$ of $P$ under the map $X_{G(n)}\rightarro X_B$ is $\Q$-rational. The inclusion of subgroups $B\subset B_1(n)$ induces a map $X_1(n)\rightarrow  X_B$.

The minimal degree of a point on $X_1(n)$ corresponding to $E$ is the minimal index of $B:=c(B_1(n))\cap G(n)$ in $G(n)$, where $c$ denotes conjugation in $\GL_2(\Z /n \Z)$.
Call this (minimal) index $d$ and let $f:X_{B}\rightarrow X_{G(n)}$ be the corresponding map of modular curves. By construction, all the points in $f^{-1}(P)$ are of degree $d$ and obviously $\gon_\Q X_{B}\leq d.$ Let $P'\in f^{-1}(P) \subseteq X_{B}$ be a point corresponding to $E$, and let $g:X_{B}\rightarrow X_1(n)$ be the map induced by the inclusion $B_1(n) \subseteq B$. Then by the argument above we have that $g(P')$ is of degree $d$ (as this is the minimal degree of a point on $X_1(n)$ corresponding to $E$), and on the other hand we have that $\gon_\Q X_1(n)\leq d$ by \cite[Proposition A.1 (vii)]{Poonen:gonality}.
\end{proof}




\vspace{20 mm}
\bibliographystyle{alpha}
\bibliography{bibfile}

\end{document}
