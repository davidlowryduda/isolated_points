\documentclass[11pt,reqno]{amsart}
% We ain't got no time for eq. nums. on the left

% Enable UTF-8 encodings for input, to enter é instead of \'{e}.
\usepackage[utf8]{inputenc}

\usepackage{amsmath,amsthm,amssymb}
% \usepackage{algorithm2e}

% Presented to you by Technicolor, and the number 3
\usepackage{graphics}
\usepackage{hyperref}
\usepackage[usenames, dvipsnames]{xcolor}

% For full page usage, shockingly
\usepackage{fullpage}

% Don't worry about starred environments. YOU are the star!
\usepackage{mathtools}
\mathtoolsset{showonlyrefs}

% For ease in writing labels and references
%\usepackage{showkeys}
\usepackage[square,sort,comma,numbers]{natbib}

% For pretty hyperlinks
\definecolor{darkblue}{rgb}{0.0,0.0,0.3}
\hypersetup{colorlinks,breaklinks,
  linkcolor=darkblue,urlcolor=darkblue,
anchorcolor=darkblue,citecolor=darkblue}

\theoremstyle{plain}
\newtheorem{theorem}{Theorem}%[section]
\newtheorem*{theorem*}{Theorem}
\newtheorem{lemma}[theorem]{Lemma}
\newtheorem{proposition}[theorem]{Proposition}
\newtheorem*{proposition*}{Proposition}
\newtheorem{corollary}[theorem]{Corollary}
\newtheorem*{corollary*}{Corollary}
\newtheorem{claim}[theorem]{Claim}
\newtheorem{conjecture}[theorem]{Conjecture}
\newtheorem{question}[theorem]{Question}
\theoremstyle{definition}
\newtheorem{remark}[theorem]{Remark}
\newtheorem{definition}[theorem]{Definition}
\newtheorem{example}[theorem]{Example}
\newtheorem{exercise}[theorem]{Exercise}
\newtheorem{algorithm}[theorem]{Algorithm}
\DeclareMathOperator{\gon}{gon}
%\numberwithin{equation}{section}

\renewcommand{\epsilon}{\varepsilon}
\renewcommand{\phi}{\varphi}
\renewcommand{\theta}{\vartheta}
\newcommand{\Supp}{\operatorname{Supp}}
\renewcommand\labelenumi{(\theenumi)}
\renewcommand{\theenumi}{\roman{enumi}}

%%% Basic Macro%%%%%%%%%%%%%%%%%%%%%%%%%%%%%%%%%%%%%%%%
\def\ol#1{\overline{#1}}% 		overline
\def\wh#1{\widehat{#1}}% 	wide hat
\def\wt#1{\widetilde{#1}}% 	wide tilde
\def\ul#1{\underline{#1}}% 	underline
\def\smcompactification#1{\ol{#1}}% 	wide tilde
%%% Define \Alphabet&\endpiece------------------------------------------------------------------------
\def\Alphabet{A,B,C,D,E,F,G,H,I,J,K,L,M,N,O,P,Q,R,S,T,U,V,W,X,Y,Z}%  Capitalized Alphabet
\def\alphabet{a,b,c,d,e,f,g,h,i,j,k,l,m,n,o,p,q,r,s,t,u,v,w,x,y,z}%	lowercase alphabet
\def\endpiece{xxx}%									marks end of list
%%% Define \makeAlphabet------------------------------------------------------------------------
\def\makeAlphabet[#1]{\expandafter\makeA#1,xxx,}%		Ex. \makeAlphabet[A,B]
\def\makealphabet[#1]{\expandafter\makea#1,xxx,}%		Ex. \makealphabet[c,d]
\def\makeA#1,{\def\temp{#1}\ifx\temp\endpiece\else%
	\mkbb{#1}\mkfrak{#1}\mkbf{#1}\mkcal{#1}\mkscr{#1}\mkbs{#1}\expandafter\makeA\fi}%
\def\makea#1,{\def\temp{#1}\ifx\temp\endpiece\else\mkfrak{#1}\mkbf{#1}\mkbs{#1}\expandafter\makea\fi}%
\def\mkbb#1{\expandafter\def\csname bb#1\endcsname{\mathbb{#1}}}%      Define bb
\def\mkfrak#1{\expandafter\def\csname fr#1\endcsname{\mathfrak{#1}}}%    Define frak
\def\mkbf#1{\expandafter\def\csname b#1\endcsname{\mathbf{#1}}}%           Define bold letters
\def\mkcal#1{\expandafter\def\csname c#1\endcsname{\mathcal{#1}}}%       Define calligraphy
\def\mkscr#1{\expandafter\def\csname s#1\endcsname{\mathscr{#1}}}%       Define script
\def\mkbs#1{\expandafter\def\csname bs#1\endcsname{{\boldsymbol{#1}}}}%       Define bold symbol
%%% Define \makeop-------------------------------------------------------------------------------------------------------
\def\makeop[#1]{\xmakeop#1,xxx,}%					Ex. \makeop[Hom,Spec]
\def\mkop#1{\expandafter\def\csname #1\endcsname{{\mathrm{#1}}}} %
\def\xmakeop#1,{\def\temp{#1}\ifx\temp\endpiece\else\mkop{#1}\expandafter\xmakeop\fi}%
\def\makeup[#1]{\xmakeup#1,xxx,}%					Ex. \makeop[Hom,Spec]
\def\mkup#1{\expandafter\def\csname #1\endcsname{{\mathrm{#1}\,}}} %
\def\xmakeup#1,{\def\temp{#1}\ifx\temp\endpiece\else\mkup{#1}\expandafter\xmakeup\fi}%
%%% Initialize------------------------------------------------------------------------------------------------------------------
% Define Alphabets.  Alphabets stored in \Alphabet
\makeAlphabet[\Alphabet]%				Define bb, frak, bf, cal for Capitalized Alphabet
\makealphabet[\alphabet]%  				Define frak and bf for uncapitalized alphabet
% Define Operators.  Separate Items by using comma.
\makeop[Hom,Aut,End,Mor,SL,GL,H,ord,Irr,Ell,Gal,Cl,Pic,NS,Gal,d,Re,Im,res,Symb,Ev,Char,Ram,SU]
% 		Homs
\makeup[Spec,Proj,id,dR,new,old,AJ,tr,dim,ker,im,coker]

% Nongross real and imaginary parts
\renewcommand{\Im}{\operatorname{Im}}
\renewcommand{\Re}{\operatorname{Re}}
\newcommand{\Q}{\bQ}
\newcommand{\Z}{\bZ}
\newcommand{\PP}{\mathbb P}
\newcommand{\filip}[1]{{\color{cyan} \textsf{$\sun\sun\sun$ Filip: [#1]}}}
\newcommand{\davidnote}[1]{\marginpar{\footnotesize{\textcolor{Plum}{David: #1}}}}

% Don't have subsections appear in TOC
%\setcounter{tocdepth}{1}

\title{Towards a Classification of Sporadic $j$-invariants}
\author[Doe]{Your Name Here}
\address{
  John Doe,
  Monsters University,
  Department of Mathematics,
  123 Main Street,
  Anywhere, CA
  United States of America
}
\email{\url{foobar@example.com}}
\thanks{JD is supported by the Imaginary Friend association}

\author[Bourdon]{Abbey Bourdon}
\address{
  Abbey Bourdon,
  Wake Forest University,
  Department of Mathematics, 127 Manchester Hall, PO Box 7388, Winston-Salem, NC 27109
}
\email{\url{bourdoam@wfu.edu}}
\thanks{AB is supported by NSF Grant DMS-2145270.}

\author[Hashimoto]{Sachi Hashimoto}
\address{%
  Sachi Hashimoto,
  Max Planck Institut für Mathematik in den Naturwissenschaften,
  Inselstraße 22,
  04103 Leipzig
}
\email{\url{sachi.hashimoto@mis.mpi.de}}
\urladdr{\url{sachihashimoto.github.io/}}


\author[Keller]{Timo Keller}
\address{Timo Keller, Leibniz Universität Hannover, Institut für Algebra, Zahlentheorie und Diskrete Mathematik, Welfengarten 1, 30167 Hannover, Germany}
\email{keller@math.uni-hannover.de}
\urladdr{\url{https://www.timo-keller.de}}


\author[Lowry-Duda]{David Lowry-Duda}
\address{%
  David Lowry-Duda, ICERM, 121 South Main Street, Box E, 11th Floor,
  Providence, RI, 02903
}
\email{\url{david@lowryduda.com}}
\urladdr{\url{https://davidlowryduda.com}}
\thanks{%
  DLD was supported by the Simons Collaboration in Arithmetic Geometry, Number
  Theory, and Computation via the Simons Foundation grant 546235.
}

\author[Shukla]{Himanshu Shukla}
\address{Himanshu Shukla, Mathematisches Institut, Uiversit\"{a}t Bayreuth, Universit\"{a}tstrasse 30, 95444 Bayreuth, Germany}
\email{Himanshu.Shukla@uni-bayreuth.de}
\urladdr{\url{https://www.mathe2.uni-bayreuth.de/hishukla/}}
\thanks{HS is supported by the DFG-grant STO 299/17-1}

\date{\today}

\begin{document}
\begin{abstract}
We develop an algorithm to test whether a non-CM elliptic curve $E/\Q$ gives rise to a sporadic point of any degree on any modular curve of the form $X_1(N)$. This builds on prior work of Zywina which gives a method for computing the image of the adelic Galois representation associated to $E$. Running this algorithm on all elliptic curves presetly in the LMFDB yields strong evidence for the fact that $E$ gives rise to a sporadic point if and only if $j(E)=-140625/8$ or $-9317$.
    \end{abstract}
\maketitle


The modular curve $X_1(N)$ is an algebraic curve over $\Q$ whose non-cuspidal points parametrize elliptic curves with a distinguished point of order $N$. We are interested in studying \textbf{sporadic} points $x \in X_1(N)$, which are points for which there are only finitely many points on $X_1(N)$ of degree at most $\deg(x)$. Hence a non-cuspidal sporadic point corresponds to an elliptic curve with a point of order $N$ in ``usually low degree." Elliptic curves with complex multiplication (CM) provide many natural examples of sporadic points, as the extra endomorphisms of a CM elliptic curve give constraints on the size of the image of the associated Galois representation. Indeed, there exist sporadic CM points on $X_1(N)$ for all $N\geq 721$; see \cite[Theorem 8.2]{CGPS2022}.

Non-CM sporadic points on $X_1(N)$ remain much more mysterious. One recent line of investigation has focused on the class of sporadic points associated to non-CM elliptic curves with $j$-invariant in $\Q$. To date, there are only two known examples of such curves, up to isomorphism over $\overline{\Q}$:
\begin{itemize}
\item The elliptic curve with $j$-invariant $-140625/8$ corresponds to points of degree 3 on $X_1(21)$. As there are only finitely many points of degree at most 3 on this modular curve, these are sporadic. This example was first discovered by Najman \cite{najman16}. In fact, this is the unique elliptic curve giving a sporadic point of degree at most 3 on \emph{any} modular curve of the form $X_1(N)$, as shown in recent work of Derickx, Etropolski, van Hoeij, Morrow, and Zureick-Brown \cite{DEvHMZB2021}.
\item The elliptic curve with $j$-invariant $-9317$ gives a degree 6 point on $X_1(37)$, as in work of van Hoeij \cite{vanHoeij}. Since this degree is less than half the $\Q$-gonality of $X_1(37)$, as computed in \cite{DerickxVanHoeij2014}, the point is necessarily sporadic by work of Frey \cite{frey}.
\end{itemize}

We say these are \textbf{sporadic $j$-invariants} since they are the image of a sporadic point on $X_1(N)$. We have good reason to believe that the collection of all sporadic $j$-invariants in $\Q$ is finite.
\begin{theorem}[Bourdon, Ejder, Liu, Odumodu, Viray \cite{BELOV}]
Suppose there exists a constant $C$ such that the mod $\ell$ Galois representation of any non-CM elliptic curve over $\Q$ is surjective for primes $\ell>C$. Then there are only finitely many sporadic $j$-invariants in $\Q$.
\end{theorem}

\noindent The assumption about Galois representations stated in Theorem 1 was originally asked as a question by Serre \cite{serre72}, and it has now been formally conjectured by both Sutherland \cite{sutherland} and Zywina \cite{ZywinaImages}. It is even suspected that $C=37$. Theorem 1 is known unconditionally for points of odd degree \cite{OddDeg}: indeed, $j=-140625/8$ is the only non-CM $j$-invariant in $\Q$ giving a sporadic point of odd degree on $X_1(N)$, even as $N$ ranges over all positive integers. 

In \cite{BELOV}, they ask whether one can explicitly determine the set of sporadic $j$-invariants in $\Q$, and this is the motivation for the present work. In particular, we conjecture that the two non-CM sporadic $j$-invariants identified above are in fact the \emph{only} sporadic $j$-invariants in $\Q$ associated to non-CM elliptic curves. 

\begin{conjecture}
If $x\in X_1(N)$ is a sporadic point with $j(x) \in \Q$, then $j(x)=-140625/8$ or $-9317$ or is one of the 13 CM $j$-invariants in $\Q$. 
\end{conjecture}

\noindent Since any CM elliptic curve is known to produce sporadic points on infinitely many modular curves of the form $X_1(N)$, it follows conversely that every $j$-invariant in this set is sporadic.

The basis for this conjecture is the following result.

\begin{theorem}
Let $x=[E,P]\in X_1(N)$ be a non-CM sporadic point with $j(E) \in \Q$.
Fix an equation for $E/\Q$ and let $N_E$ denote its conductor.
Suppose that one of the following holds:
\begin{itemize}
    \item$N_E \leq 500{,}000$,
    \item $N_E$ is only divisible by primes $p \leq 7$, or
    \item $N_E=p \leq 200{,}000{,}000$ for some prime number $p$.
\end{itemize}
Then $j(E) =-140625/8$ or $-9317$.
\end{theorem}

Since both known examples of sporadic points on $X_1(N)$ associated to non-CM rational $j$-invariants lie above exceptional rational points on $X_0(N)$ -- that is, they correspond to rational points in cases where the set of all such rational points is finite -- is natural to ask whether this construction might yield other examples of sporadic points. Our work shows that the answer is no.
\begin{corollary}
Let $X_0(N)$ have genus greater than 0, and let $E$ be a non-CM elliptic curve corresponding to a rational point on $X_0(N)$. If there exists a sporadic point $x \in X_1(N')$ with $j(x)=j(E)$, then $j(E) =-140625/8$ or $-9317$.
\end{corollary}

\section{Proving that a point is not sporadic}

\begin{algorithm}
\textbf{Input:} A non-CM $j$-invariant $j = j(E) \in \Q$.

\textbf{Output:} \textsc{False} if $j$ is not a sporadic $j$-invariant.

\begin{enumerate}
    \item Compute the adelic image $G$ of $E/\Q$ as a subgroup of $\GL_2(\hat{\Z})$ using Zywina's algorithm. Represent the output as some \emph{level} $M$ together with a subgroup $G$ of $\GL_2(\Z/M\Z)$.

    \item Reduce level.

    \item For all $12 < b \mid M$ compute the minimum degree of the field of definition of a point with respect to $G \mod b$, the image of $G$ in $\GL_2(\Z/b\Z)$. (For that, compute the orbits of $\langle G, -G \rangle \mod b$ on the elements of $(\Z/b\Z)^2$ of order $b$. Sort the orbits by size. Return ) 
    %\filip{We need to look at the orbits of $\langle G, -G \rangle \mod b$ on $(\Z/b\Z)^2$, not $\bP^1(\Z/b\Z)$. If we consider the action on $\bP^1(\Z/b\Z)$, the orbits might be too small and hence the degree we get. Furthermore, it is necessary to consider only the elements of $(\Z/b\Z)^2$ of order $b$.}
\end{enumerate}
\end{algorithm}

(In the \texttt{Magma} implementation, we have to be careful that \texttt{Magma} uses right actions, but in our mathematics, we use left action. We therefore use the transposes of the matrices in $\GL_2(\Z/M\Z)$.)


\section{Reduce Level}
\noindent \textbf{Input:} Non-CM elliptic curve $E/\Q$\\
\textbf{Output:} Finite set of positive integers $\mathcal{I}_1$ such that any isolated point $x \in X_1(n)$ with $j(x) =j(E)$ maps under the natural projection map to an isolated point on $X_1(d)$ for $d \in \mathcal{I}_1$.
\begin{theorem}
Let $E/\Q$ be a non-CM elliptic curve, and let $\rho_E$ denote the adelic Galois representation of $E$. Let $M_E$ be any positive integer such that 
\[
\im \rho_E=\pi^{-1}(\im \rho_{E, M_E}).
\]
Then for any $x \in X_1(n)$ with $j(x)=j(E)$ we have $\deg(x)=\deg(f)\cdot \deg(f(x))$, where $f: X_1(n) \rightarrow X_1(\gcd(n,M_E))$ denotes the natural map. It follows that:
\begin{enumerate}
\item If $x$ is $\mathbb{P}^1$-isolated, then $f(x)$ is $\mathbb{P}^1$-isolated.
\item If $x$ is AV-isolated, then $f(x)$ is AV-isolated.
\item If $x$ is sporadic, then $f(x)$ is sporadic.
\end{enumerate}
\end{theorem}

\begin{proof}
This follows from the arguments in Section 5.3 of BELOV (see Remark 5.5 in particular) and Theorem 4.3.
\end{proof}

Let $S_E=\{2,3\} \cup \{\ell : \rho_{E,\ell^{\infty}}(\Gal_{\Q}) \neq \GL_2(\Z_{\ell})\}$. Let $M_E=M_1\cdot M_2$ where $\Supp(M_1) \subseteq S_E$ and $\Supp(M_2) \cap S_E =\emptyset$.

\begin{corollary}
Let $E/\Q$ be a non-CM elliptic curve, and let $\rho_E$ denote the adelic Galois representation of $E$. 
Then for any $x \in X_1(n)$ with $j(x)=j(E)$ we have $\deg(x)=\deg(f)\cdot \deg(f(x))$, where $f: X_1(n) \rightarrow X_1(\gcd(n,M_1))$ denotes the natural map. It follows that:
\begin{enumerate}
\item If $x$ is $\mathbb{P}^1$-isolated, then $f(x)$ is $\mathbb{P}^1$-isolated.
\item If $x$ is AV-isolated, then $f(x)$ is AV-isolated.
\item If $x$ is sporadic, then $f(x)$ is sporadic.
\end{enumerate}
\end{corollary}

\begin{proof}
This should follow from BELOV Proposition 5.7.
\end{proof}

Main idea beyond this: Let $m$ be the product of the primes in $\Supp(M_1)$. By Proposition 5.8 in BELOV, it suffices to replace $M_1$ in the corollary above with the level of the $m$-adic Galois representation. This should be the main point of the ``ReducedLevel" operation: to compute this level. Note in particular we do not need to add in 2 and 3 if they were not already in $\Supp(M_E)$.

\section{Compute Degree}

\noindent \textbf{Input:} $\mathcal{I}_1$ from step 1\\
\textbf{Output:} For each $d \in \mathcal{I}_1$, compute degree(s) of closed point on $X_1(n)$ associated to $E$. Return ???

\begin{proposition}
Let $E/\Q$ be a non-CM elliptic curve and $n \in \mathbb{Z}^+$. If $\im \rho_{E,n} \cong G \leq \GL_2(\Z/n\Z)$, then ...
\end{proposition}



\section{Riemann-Roch}

\noindent \textbf{Input:} Output from step 2\\
\textbf{Output:} Smaller list of pairs to check

\begin{lemma}
Let $C$ be a curve over a number field $k$ with genus $g$. If $x \in C(\overline{k})$ is a closed point with degree
$\geq g+1$, then $x$ is not $\mathbb{P}^{1}$-isolated.
\end{lemma}

\begin{proof}
This is Lemma 13 in revision of B., Gill, Rouse, Watson.
\end{proof}

\section{Remaining Cases}

Filip's arguemnts



\section{Genus 0 adelic images do not produce isolated points}

Let $E/ \Q$ be an elliptic curve and $G\in \GL_2(\hat \Z)$ its adelic image. By $G(n)$ denote the image of its mod $n$ representation. Denote by $B_1(n)$ the subgroup of $\GL_2(\Z /n \Z)$ consisting of the upper triangular matrices with a $1$ in the upper left entry. Note that $X_{B_1(n)}=X_1(n)$. We say that a congruence group $\Gamma$ is of genus $g$ if $X_\Gamma$ is of genus $g$. We say that a point $P$ corresponds to an elliptic curve $E$ if $j(P)=j(E)$.

\begin{proposition}
Let $E/\Q$ be an elliptic curve with adelic image of genus $0$. Then there does not exist an $n$ such that a point corresponding to $E$ on $X_1(n)$ is sporadic.
\end{proposition}
\begin{proof}
Since $G$ is by assumption of genus $0$, it follows that so is $G(n)$. Let $P$ be a $\Q$-rational point on $X_{G(n)}$ corresponding to $E$. 

%Let $B$ be the group generated by $G(n)$ and a conjugate of $B_1(n)$ in $\GL_2(\Z/n\Z)$; this is again a group of genus $0$ on which the image $P'$ of $P$ under the map $X_{G(n)}\rightarro X_B$ is $\Q$-rational. The inclusion of subgroups $B\subset B_1(n)$ induces a map $X_1(n)\rightarrow  X_B$. 

The minimal degree of a point on $X_1(n)$ corresponding to $E$ is the minimal index of $B:=c(B_1(n))\cap G(n)$ in $G(n)$, where $c$ denotes conjugation in $\GL_2(\Z /n \Z)$. 
Call this (minimal) index $d$ and let $f:X_{B}\rightarrow X_{G(n)}$ be the corresponding map of modular curves. By construction, all the points in $f^{-1}(P)$ are of degree $d$ and obviously $\gon_\Q X_{B}\leq d.$ Let $P'\in f^{-1}(P) \subseteq X_{B}$ be a point corresponding to $E$, and let $g:X_{B}\rightarrow X_1(n)$ be the map induced by the inclusion $B_1(n) \subseteq B$. Then by the argument above we have that $g(P')$ is of degree $d$ (as this is the minimal degree of a point on $X_1(n)$ corresponding to $E$), and on the other hand we have that $\gon_\Q X_1(n)\leq d$ by \cite[Proposition A.1 (vii)]{Poonen:gonality}.
\end{proof}


\vspace{20 mm}
\bibliographystyle{alpha}
\bibliography{bibfile}

\end{document}
